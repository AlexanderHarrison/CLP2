%
% Copyright 2018 Joel Feldman, Andrew Rechnitzer and Elyse Yeager.
% This work is licensed under a Creative Commons Attribution-NonCommercial-ShareAlike 4.0 International License.
% https://creativecommons.org/licenses/by-nc-sa/4.0/
%
\questionheader{ex:s3.4}

%%%%%%%%%%%%%%%%%%
\subsection*{\Conceptual}
%%%%%%%%%%%%%%%%%%

\begin{question}[2015A]
Decide whether the following statement is true or false.
If false, provide a counterexample. If true provide a brief justification.

\begin{enumerate}[(a)]
\item []\color{blue}
If $\displaystyle\sum_{n=1}^{\infty}(-1)^{n+1}b_n$ converges, then $\displaystyle\sum_{n=1}^{\infty} b_n$ also converges.
\end{enumerate}
\end{question}

\begin{hint}
What is conditional convergence?
\end{hint}

\begin{answer}
False. For example, $b_n=\frac{1}{n}$ provides a counterexample.
\end{answer}

\begin{solution}
False. For example if $b_n=\frac{1}{n}$, then
$\sum\limits_{n=1}^{\infty}(-1)^{n+1}b_n
   =\sum\limits_{n=1}^{\infty}(-1)^{n+1}\frac{1}{n}$
converges by the alternating series test, but
$\sum\limits_{n=1}^{\infty} \frac{1}{n}$ diverges by the $p$--test.

Remark: if we had added that $\{b_n\}$ is a sequence of \emph{alternating terms}, then by Theorem~\eref{CLP101}{thm:SRabs} in the CLP--II text, the statement would have been true. This is because $\displaystyle\sum_{n=1}^{\infty}(-1)^{n+1}b_n$ would either be equal to $\displaystyle\sum_{n=1}^{\infty} |b_n|$ or $-\displaystyle\sum_{n=1}^{\infty} |b_n|$.
\end{solution}

%%%%%%%%%%%%%%%%%%%
\begin{Mquestion}
Describe the sequence $\displaystyle\sum_{n=1}^\infty a_n$ based on whether
$\displaystyle\sum_{n=1}^\infty a_n$ and $\displaystyle\sum_{n=1}^\infty |a_n|$ converge or diverge, using vocabulary from this section where possible.
\begin{center}
\begin{tabular}{|l|c|c|}
\hline
& \textbf{$\sum\limits^{\vphantom{{\frac12}}}  a_n$ converges} & \textbf{$\sum a_n$ diverges}\\[10pt]
\hline
 \textbf{$\sum\limits^{\vphantom{{\frac12}}}  |a_n|$ converges}&&\\[10pt]
 \hline
  \textbf{$\sum\limits^{\vphantom{{\frac12}}}  |a_n|$ diverges}&&\\[10pt]
  \hline
\end{tabular}
\end{center}

\end{Mquestion}
\begin{hint}
If $\sum |a_n|$ converges, then $\sum a_n$ is guaranteed to converge as well. \\
(That's Theorem~\eref{CLP101}{thm:SRabs} in the CLP--II text.) So, one of the blank spaces describes an impossible sequence.
\end{hint}
\begin{answer}
\begin{center}
\begin{tabular}{|l|c|c|}
\hline
& \textbf{$\sum a_n$ converges} & \textbf{$\sum\vphantom{^{\frac12}} a_n$ diverges}\\[10pt]
\hline
 \textbf{$\sum\vphantom{^{\frac12}}  |a_n|$ converges}&converges absolutely&not possible\\[10pt]
 \hline
  \textbf{$\sum\vphantom{^{\frac12}}  |a_n|$ diverges}&converges conditionally&diverges\\[10pt]
  \hline
\end{tabular}
\end{center}
\end{answer}
\begin{solution}
Absolute convergence describes the situation where $\sum |a_n|$ converges (see Definition~\eref{CLP101}{def:SRabsCond} in the CLP--II text). By Theorem~\eref{CLP101}{thm:SRabs} in the CLP--II text, this guarantees that also $\sum a_n$ converges.

Conditional convergence describes the situation where $\sum |a_n|$ diverges but $\sum a_n$ converges (see again Definition~\eref{CLP101}{def:SRabsCond} in the CLP--II text).

If $\sum a_n$ diverges, we just say it diverges. The reason is that if $\sum a_n$ diverges, we automatically know $\sum |a_n|$ diverges as well, so there's no need for a special name.
\begin{center}
\begin{tabular}{|l|c|c|}
\hline
& \textbf{$\sum a_n$ converges} & \textbf{$\sum\vphantom{^{\frac12}} a_n$ diverges}\\[10pt]
\hline
 \textbf{$\sum\vphantom{^{\frac12}}  |a_n|$ converges}&converges absolutely&not possible\\[10pt]
 \hline
  \textbf{$\sum\vphantom{^{\frac12}}  |a_n|$ diverges}&converges conditionally&diverges\\[10pt]
  \hline
\end{tabular}
\end{center}

\end{solution}
%%%%%%%%%%%%%%%%%%%




%%%%%%%%%%%%%%%%%%
\subsection*{\Procedural}
%%%%%%%%%%%%%%%%%%

\begin{Mquestion}[2015A]
Determine whether the series $\displaystyle\sum_{n=1}^{\infty}\displaystyle\frac{(-1)^n}{9n+5}$ is absolutely convergent, conditionally convergent, or divergent; justify your answer.
\end{Mquestion}

%\begin{hint}
%\end{hint}

\begin{answer}
conditionally convergent
\end{answer}

\begin{solution}
The series $\sum\limits_{n=1}^{\infty}\frac{(-1)^n}{9n+5}$
converges by the alternating series test. On the other hand the
series $\sum\limits_{n=1}^{\infty}\big|\frac{(-1)^n}{9n+5}\big|
=\sum_{n=1}^{\infty}\frac{1}{9n+5}$ diverges by the limit
comparison test with $b_n=\frac{1}{n}$. So the given series is
conditionally convergent.
\end{solution}
%%%%%%%%%%%%%%%%%%%

\begin{Mquestion}[2016Q6]
Determine whether the series
$\displaystyle\sum_{n=1}^\infty\frac{(-1)^{2n+1}}{1+n}$
is absolutely convergent, conditionally convergent, or divergent.
\end{Mquestion}

\begin{hint}
Be careful about the signs.
\end{hint}

\begin{answer}
The series diverges.
\end{answer}

\begin{solution}
Note that $(-1)^{2n+1}=(-1)\cdot(-1)^{2n}=-1$. So we can simplify
\begin{align*}
  \sum_{n=1}^\infty\frac{(-1)^{2n+1}}{1+n}=-\sum_{n=1}^\infty\frac{1}{1+n}
\end{align*}
Since $\displaystyle \frac{1}{1+n} \geq \frac{1}{n+n} = \frac{1}{2n}$,
$\ \ \displaystyle \sum_{n=1}^\infty\frac{1}{1+n}$
diverges by the comparison test with the divergent harmonic series $\sum\limits_{n=1}^\infty\frac{1}{n}$. The extra overall factor of $-1$ in the original series
does not change the conclusion of divergence.

\end{solution}
%%%%%%%%%%%%%%%%%%%

\begin{question}[2016Q6]
The series
$\displaystyle \sum_{n=1}^{\infty} (-1)^{n-1} \frac{1+4^n}{3+2^{2n}}$ either:
              converges absolutely;
             converges conditionally;
            diverges; or
            none of the above.
Determine which  is correct.

\end{question}

\begin{hint}
Does the alternating series test really apply?
\end{hint}

\begin{answer}
It diverges.
\end{answer}

\begin{solution}
Since
\begin{align*}
 \lim_{n\to\infty} \frac{1+4^n}{3+2^{2n}}
=\lim_{n\to\infty} \frac{1+4^n}{3+4^n}
= 1
\end{align*}
the alternating series test cannot be used. Indeed,
$\displaystyle \lim_{n\to\infty} (-1)^{n-1} \frac{1+4^n}{3+2^{2n}}$ does not exist
(for very large $n$,   $ (-1)^{n-1} \frac{1+4^n}{3+2^{2n}}$ alternates between a number
close to $+1$ and a number close to $-1$)
so the divergence test says that the series diverges.
(Note that ``none of the above'' cannot possibly be the correct answer ---
\emph{every} series either converges absolutely, converges conditionally, or diverges.)
\end{solution}
%%%%%%%%%%%%%%%%%%%



\begin{Mquestion}[2016Q5]
Does the series $\displaystyle \sum_{n=5}^\infty \frac{\sqrt{n}\cos n}{n^2-1}$
converge conditionally, converge absolutely, or diverge?
\end{Mquestion}

\begin{hint}
What does the summand look like when $n$ is very large?
\end{hint}

\begin{answer}
It converges absolutely.
\end{answer}

\begin{solution}
First, we'll develop some intuition. For very large $n$
\begin{align*}
\left|\frac{\sqrt{n}\cos(n)}{n^2-1}\right|
&\approx \left|\frac{\sqrt{n}\cos(n)}{n^2}\right|
= \left|\frac{\cos(n)}{n^{3/2}}\right|
\le \frac{1}{n^{3/2}}
\end{align*}
since $\left|\cos(n)\right| \leq 1$ for all $n$. By the $p$--test, the series
$\displaystyle \sum_{n=5}^\infty \frac{1}{n^p}$ converges for all $p>1$. So we would
expect the given series to converge absolutely.

Now, to confirm that our intuition is correct, we'll first try setting
$a_n=\left| \frac{\sqrt{n}\cos(n)}{n^2-1}\right|$
and $b_n = \frac{1}{n^{3/2}}$.
\begin{align*}
\lim_{n \to \infty}\frac{a_n}{b_n}&=\lim_{n \to \infty} \frac{\left|\frac{\sqrt{n}\cos(n)}{n^2-1}\right| }{\frac{1}{n^{3/2}}} = \lim_{n \to \infty}\frac{\sqrt{n}\cdot\sqrt{n}^3|\cos n|}{n^2-1}\\
&=\lim_{n \to \infty}\frac{n^2|\cos n|}{n^2-1} = \lim_{n \to \infty}\left(\frac{n^2}{n^2-1}\right)|\cos n|\\
& = \lim_{n \to \infty}1\cdot|\cos n|
\end{align*}
Unfortunately, this limit doesn't exist, so we can't use the limit comparison theorem. We'll need a slightly different tactic.

\begin{description}
\item[Solution 1:] Let's try the limit comparison test with a different $p$--series. The question is, which one. Let's start by leaving $p$ as a variable, then see what happens.

We  apply the limit comparison test with  $a_n=\left| \frac{\sqrt{n}\cos(n)}{n^2-1}\right|$
and  $b_n=\frac{1}{n^p}$. We'll choose a specific $p$ shortly.
\begin{align*}
\lim_{n\to\infty} \frac{a_n}{b_n}
&=\lim_{n\to\infty} \frac{\big|\sqrt n\,\cos n/(n^2-1)\big|}{1/n^p} \\
&=\lim_{n\to\infty} \frac{n^{p+0.5}|\cos n|}
                                   {n^2(1-1/n^2)}
= \lim_{n\to\infty} \frac{|\cos n|}{n^{(3/2-p)}(1-1/n^2)} \\
&=0\quad\text{ if $p<\frac{3}{2}$}
\end{align*}
the limit comparison test says that if $p<\frac{3}{2}$ and the series
$\sum\limits_{n=5}^\infty b_n$ converges (which is the case if $p>1$) then the series
$\sum\limits_{n=5}^\infty  \left|\frac{\sqrt{n}\cos(n)}{n^2-1}\right|$ also converges.
So choosing any $1<p<\frac{3}{2}$, for example $p=\frac{5}{4}$, we conclude that
the given series converges absolutely.

\item[Solution 2:]
Let's try to use the direct comparison test. When we were trying to develop intuition, we noticed the following:
\begin{align*}
\left|\frac{\sqrt{n}\cos(n)}{n^2-1}\right|
&\approx \left|\frac{\sqrt{n}\cos(n)}{n^2}\right|
= \left|\frac{\cos(n)}{n^{3/2}}\right|
\le \frac{1}{n^{3/2}}
\end{align*}
It's not the case that our terms are less than $\frac{1}{n^{3/2}}$, but perhaps they would be less than, say $\frac{2}{n^{3/2}}$. Let's trace our reasoning above backwards.
\begin{align*}
\frac{2}{n^{3/2}} &\geq \left|\frac{2\cos n}{n^{3/2}}\right|
=\left|\frac{\sqrt{n}\cos n}{(1/2)n^{2}}\right|
\geq \left|\frac{\sqrt{n}\cos n}{n^{2}-1}\right|
\end{align*}
where the final inequality holds for all $n \ge 2$.

Since $\sum\limits_{n=1}^\infty \frac{2}{n^{3/2}}$ converges by the $p$-test, our series  converges as well by the direct comparison test.
\end{description}
\end{solution}


\begin{Mquestion}[2012A]
Determine (with justification!) whether the series
$\displaystyle\sum_{n=1}^\infty\frac{n^2-\sin n}{n^6+n^2}$
converges absolutely, converges but not absolutely, or diverges.
\end{Mquestion}

\begin{hint}
What does the summand look like when $n$ is very large?
\end{hint}

\begin{answer}
It converges absolutely.
\end{answer}

\begin{solution}
We first develop some intuition about $\displaystyle\sum_{n=1}^\infty\left|\frac{n^2-\sin n}{n^6+n^2}\right|$, where we take the absolute value of the summands to consider whether the series converges \emph{absolutely}. For very large $n$, $n^2$ dominates
$\sin n$ and $n^6$ dominates $n^2$ so that
\begin{align*}
\left|\frac{n^2-\sin n}{n^6+n^2}\right|
\approx \frac{n^2}{n^6}
=\frac{1}{n^4}
\end{align*}
The series $\displaystyle \sum_{n=1}^{\infty} \frac{1}{n^4}$
converges by the $p$--test with $p=4>1$.
We expect the given series to converge too.

To verify that  our intuition is correct,
we apply the limit comparison test with
\begin{align*}
a_n= \frac{n^2-\sin n}{n^6+n^2} \quad\text{and}\quad b_n= \frac{1}{n^4}
\end{align*}
which is valid since
\begin{equation*}
\lim_{n\rightarrow\infty} \frac{a_n}{b_n}
=\lim_{n\rightarrow\infty}\left|\frac{(n^2-\sin n)}{n^6+n^2}\right|\cdot\frac{n^4}{1}
=\lim_{n\rightarrow\infty}\frac{|n^6-n^4\sin n|}{n^6+n^2}
=\lim_{n\rightarrow\infty}\frac{1-n^{-2}\sin n}{1+n^{-4}}
=1
\end{equation*}
exists and is nonzero. Since the series $\sum\limits_{n=1}^\infty b_n$
converges, the series $\sum\limits_{n=1}^\infty\dfrac{|n^2-\sin n|}{n^6+n^2}$ converges too. Therefore, the series $\sum\limits_{n=1}^\infty\dfrac{n^2-\sin n}{n^6+n^2}$ converges absolutely.
\end{solution}

%%%%%%%%%%%%%%%%%%%
\begin{Mquestion}[2012A]
Determine (with justification!) whether the series
$\displaystyle\sum_{n=0}^\infty\frac{(-1)^n(2n)!}{(n^2+1)(n!)^2}$
converges absolutely, converges but not absolutely, or diverges.
\end{Mquestion}

\begin{hint}
This is a trick question. Be sure to verify \emph{all} of the hypotheses
of any convergence test you apply.
\end{hint}

\begin{answer}
It diverges.
\end{answer}

\begin{solution}
You might think that this series converges by the alternating series test.
But you would be wrong. The problem is that $\{a_n\}$ does not converge to zero as $n\rightarrow\infty$, so that the series actually
diverges by the divergence test. To verify that the $n^{\text{th}}$ term does not converge to zero as $n\rightarrow\infty$ let's write
$a_n= \frac{(2n)!}{(n^2+1)(n!)^2}$ (i.e. $a_n$ is the $n^{\text{th}}$ term
without the sign) and check to see whether $a_{n+1}$ is bigger than or smaller than $a_n$.
\begin{align*}
\frac{a_{n+1}}{a_n}
&=\frac{(2n+2)!}{((n+1)^2+1)((n+1)!)^2}
\frac{(n^2+1)(n!)^2}{(2n)!}
=\frac{(2n+2)(2n+1)}{(n+1)^2}\frac{n^2+1}{(n+1)^2+1} \\
&=\frac{2(2n+1)}{(n+1)}\frac{1+1/n^2}{(1+1/n)^2+1/n^2}
=4\frac{1+1/2n}{1+1/n}\frac{1+1/n^2}{(1+1/n)^2+1/n^2}
\end{align*}
So
\begin{equation*}
\lim_{n\rightarrow\infty}\frac{a_{n+1}}{a_n}=4
\end{equation*}
and, in particular, for large $n$, $a_{n+1}>a_n$. Thus, for large $n$,
$a_n$ increases with $n$ and so cannot converge to $0$. So the series
diverges by the divergence test.
\end{solution}
%%%%%%%%%%%%%%%%%%%

\begin{Mquestion}[2012A]
Determine (with justification!) whether the series
$\displaystyle\sum_{n=2}^\infty\frac{(-1)^n}{n(\log n)^{101}}$
converges absolutely, converges but not absolutely, or diverges.
\end{Mquestion}

\begin{hint}
Try the substitution $u=\log x$.
\end{hint}

\begin{answer}
It converges absolutely.
\end{answer}

\begin{solution}
This series converges by the alternating series test. We want to know whether it converges absolutely, so we consider the seris$\displaystyle\sum_{n=2}^\infty\left|\frac{(-1)^n}{n(\log n)^{101}}\right|=\sum_{n=2}^\infty\frac{1}{n(\log n)^{101}}$.

We've seen similar function before (e.g. Example \eref{CLP101}{eg:SRlnpTest} in the CLP--II text, with $p=101>1$) and it yields nicely to the integral test. Let $f(x) = \frac{1}{x(\log x)^{101}}$. Note $f(x)$ is positive and decreasing for $n \ge 3$. Then by the integral test, the series
$\sum_{n=2}^\infty\frac{1}{n(\log n)^{101}}$ converges if and only if the integral
$\int_2^\infty \frac{1}{x(\log x)^{101}}\dee{x}$ does. We evaluate the integral using the substitution $u=\log x$, $\dee{u}=\frac{1}{x}\,\dee{x}$.
\begin{align*}
\int_2^\infty \frac{1}{x(\log x)^{101}}\dee{x}&=\lim_{b \to \infty}\int_2^b \frac{1}{x(\log x)^{101}}\dee{x}\\
&=\lim_{b \to \infty}\int_{\log 2}^{\log b}\frac{1}{u^{101}}\,\dee{u}\\
&=\lim_{b \to \infty}\left[\frac{-1}{100 u^{100}}\right]_{\log 2}^{\log b}\\
&=\frac{1}{100(\log 2)^{100}}
\end{align*}
Since the integral converges, the series
$\sum\limits_{n=2}^\infty\frac{1}{n(\log n)^{101}}$ converges, and therefore the series
$\sum\limits_{n=2}^\infty\frac{(-1)^n}{n(\log n)^{101}}$ converges absolutely.
\end{solution}

%%%%%%%%%%%%%%%%%%%%%%%%%%%%%%%%%%%%%%
\begin{question}\label{nonalt1}
Show that the series
$\displaystyle\sum_{n=1}^\infty \dfrac{\sin n}{n^2}$ converges.
\end{question}
\begin{hint}
Show that it converges absolutely.
\end{hint}
\begin{answer}
See solution.
\end{answer}
\begin{solution}
The sequence has some positive terms and some negative terms, which limits the tests we can use. However, if we consider the series $\displaystyle\sum_{n=1}^\infty \left|\dfrac{\sin n}{n^2}\right|$, we can use the direct comparison test.

For every $n$, $| \sin n| <1$, so $0\le \left|\dfrac{\sin n}{n^2}\right|<\dfrac{1}{n^2}$. Since
$\displaystyle\sum_{n=1}^\infty \dfrac{1}{n^2}$ converges, then by the direct comparison test, $\displaystyle\sum_{n=1}^\infty \left|\dfrac{\sin n}{n^2}\right|$ converges as well. Then $\displaystyle\sum_{n=1}^\infty \dfrac{\sin n}{n^2}$ converges absolutely-- in particular, it converges.
\end{solution}
%%%%%%%%%%%%%%%%%%%
\begin{question}
Show that the series
$\displaystyle\sum_{n=1}^\infty\left(\frac{\sin n}{4}-\frac{1}{8}\right)^n$ converges.
\end{question}
\begin{hint}
Use a similar method to Queston~\ref{nonalt1}.
\end{hint}
\begin{answer}
See solution.
\end{answer}
\begin{solution}
The terms of this series are sometimes negative (for odd values of $n$ where  $\sin n <\frac{1}{2}$) and sometimes positive. But, they are not strictly alternating, so we can't use the alternating series test. Instead, we use a direct comparison test to show the series converges absolutely.

\begin{align*}
-\frac14&\le \frac{\sin n}{4} \le \frac14\\
\Rightarrow \qquad\left(-\frac{1}{4}-\frac{1}{8}\right)&\le\left( \frac{\sin x}{4} -\frac18\right)< \left(\frac14-\frac18\right)\\
\Rightarrow \qquad -\frac{3}{8} &\le\left( \frac{\sin x}{4} -\frac18\right)< \frac{1}{8}\\
\Rightarrow \qquad 0 &\le\left| \frac{\sin x}{4} -\frac18\right|< \frac{3}{8}\\
\Rightarrow \qquad 0 &\le\left|\left( \frac{\sin x}{4} -\frac18\right)^n\right|< \left(\frac{3}{8}\right)^n
\end{align*}
Since $\displaystyle\sum_{n=1}^{\infty} \left(\frac{3}{8}\right)^n$ converges (it's a geometric sum with $|r|<1$), by the direct comparison test,
$\displaystyle\sum_{n=1}^{\infty} \left|\left( \frac{\sin x}{4} -\frac18\right)^n\right|$
converges as well.

Then $\displaystyle\sum_{n=1}^{\infty} \left( \frac{\sin x}{4} -\frac18\right)^n$ converges absolutely--and so it converges.
\end{solution}
%%%%%%%%%%%%%%%%%%%
\begin{question}\label{nonalt2}
Show that the series
$\displaystyle\sum_{n=1}^\infty\dfrac{\sin^2 n - \cos^2 n+\tfrac12}{2^n}$ converges.
\end{question}
\begin{hint}
Show it converges absolutely using a direct comparison test.
\end{hint}
\begin{answer}
See solution.
\end{answer}
\begin{solution}
The terms of this series are sometimes negative and sometimes positive. But, they are not strictly alternating, so we can't use the alternating series test. Instead, we use a direct comparison test to show the series converges absolutely.

\begin{align*}
0&\leq \sin^2 n \le 1\\
0&\leq \cos^2 n \le 1\\
\text{So,}\qquad -1&\leq \sin^2 n - \cos^2 n \le 1\\
-\frac12=\left(-1+\frac12\right)&\leq \left(\sin^2 n - \cos^2 n+\frac12\right) \le \left(1+\frac12\right)=\frac32\\
-\frac{1}{2}\cdot \frac{ 1}{2^n}&\leq \frac{\sin^2 n - \cos^2 n+\tfrac12}{2^n} \le \frac{3}{2}\cdot\frac{1}{2^n}\\
0&\leq \left|\frac{\sin^2 n - \cos^2 n+\tfrac12}{2^n}\right| \le \frac{3}{2^{n+1}}
\end{align*}

The series $\displaystyle\sum_{n=1}^\infty \frac{3}{2^{n+1}}$ converges, because it's a geometric series with $r=\frac{1}{2}$. By the direct comparison test,
$\displaystyle\sum_{n=1}^\infty \left|\frac{\sin^2 n - \cos^2 n+\tfrac12}{2^n}\right| $ converges as well. Then $\displaystyle\sum_{n=1}^\infty \frac{\sin^2 n - \cos^2 n+\tfrac12}{2^n}$ converges absolutely, so it converges.

\end{solution}

%%%%%%%%%%%%%%%%%%%


%%%%%%%%%%%%%%%%%%
\subsection*{\Application}
%%%%%%%%%%%%%%%%%%

\begin{question}[2016A]
Both parts of this question concern the series $\displaystyle S = \sum_{n=1}^\infty (-1)^{n-1}24n^2 e^{-n^3}$.
\begin{enumerate}[(a)]
\item
Show that the series $S$ converges absolutely.
\item
Suppose that you approximate the series $S$ by its fifth partial sum $S_5$. Give an upper bound for the error resulting from this approximation.
\end{enumerate}
\end{question}

\begin{hint}
For part (a), replace $n$ by $x$ in the absolute value of the summand.
Can you integrate the resulting function?
\end{hint}

\begin{answer}
(a) See the solution.
\qquad (b)
$|S - S_5 | \le  24 \times 36 e^{-6^3}$

\end{answer}

\begin{solution} (a)
\begin{description}
\item[Solution 1:]
We need to show that
$\sum\limits_{n=1}^\infty 24n^2 e^{-n^3}$ converges.
If we replace $n$ by $x$ in the summand, we get $f(x) = 24x^2 e^{-x^3}$,
which we can integate. (Just substitute $u=x^3$.)
So let's try the integral test.
First, we have to check that $f(x)$ is positive and decreasing.
It is certainly positive. To determine if it is decreasing, we compute
\begin{align*}
\diff{f}{x} = 48x e^{-x^3} - 24\times 3 x^4 e^{-x^3}
= 24x (2-3x^3) e^{-x^3}
\end{align*}
which is negative for $x\ge1$. Therefore $f(x)$ is decreasing for $x\ge1$,
and the integral test applies. The substitution $u=x^3$,
$\dee{u}=3x^2\,\,\dee{x}$, yields
\begin{align*}
\int f(x) \,\dee{x}
= \int 24x^2 e^{-x^3} \,\dee{x} = \int 8 e^{-u}\,\dee{u} = -8e^{-u} + C = -8e^{-x^3} + C.
\end{align*}
Therefore
\begin{align*}
\int_1^\infty f(x) \,\dee{x} &= \lim_{R \to \infty} \int_1^R f(x) \,\dee{x}
  = \lim_{R \to \infty} \bigg[ {-}8 e^{-x^3} \bigg]_1^R  \\
   &= \lim_{R \to \infty} ( - 8 e^{-R^3} + 8 e^{-1}  ) = 8 e^{-1}
\end{align*}
Since the integral is convergent, the series
$\sum\limits_{n=1}^\infty 24n^2 e^{-n^3}$ converges
and the series $\displaystyle \sum_{n=1}^\infty (-1)^{n-1}24n^2 e^{-n^3}$
converges absolutely.

\item[Solution 2:]
Alternatively, we can use the ratio test with $a_n=24n^2 e^{-n^3}$.
We calculate
\begin{align*}
\lim_{n\to\infty} \left|\frac{a_{n+1}}{a_n}\right|
&= \lim_{n\to\infty} \left|\frac{24 (n+1)^2e^{-(n+1)^3}}{24 n^2e^{-n^3}} \right| \\
&= \lim_{n\to\infty} \left( \frac{(n+1)^2}{n^2} \frac{e^{n^3}}{e^{(n+1)^3}} \right) \\
&= \lim_{n\to\infty} \left(1+\frac{1}{n}\right)^2 e^{-(3n^2+3n+1)} = 1\cdot0=0 < 1,
\end{align*}
and therefore the series converges absolutely.

\item[Solution 3:]
Alternatively, alternatively, we can use the limiting comparison test.
First a little intuition building. Recall that we need to show that
$\sum\limits_{n=1}^\infty 24n^2 e^{-n^3}$ converges. The $n^{\rm th}$
term in this series is
\begin{equation*}
a_n = 24n^2 e^{-n^3} =\frac{24 n^2}{e^{n^3}}
\end{equation*}
It is a ratio with both the numerator and denominator growing with $n$.
A good rule of thumb is that exponentials grow a lot faster than powers.
For example, if $n=10$ the numerator is $2400=2.4\times 10^3$
and the denominator is about $2\times 10^{434}$.
So we would guess that $a_n$ tends to zero as $n\rightarrow\infty$.
The question is ``does $a_n$ tend to zero fast enough with $n$ that our series converges?''. For example, we know that $\sum_{n=1}^\infty \frac{1}{n^2}$
converges (by the $p$--test with $p=2$). So if $a_n$ tends to zero faster
than $\frac{1}{n^2}$ does, our series will converge. So let's try the limiting convergence test with $a_n = 24n^2 e^{-n^3} =\frac{24 n^2}{e^{n^3}}$
and $b_n=\frac{1}{n^2}$.
\begin{align*}
\lim_{n\rightarrow\infty}\frac{a_n}{b_n}
=\lim_{n\rightarrow\infty}\frac{24n^2 e^{-n^3}}{1/n^2}
=\lim_{n\rightarrow\infty}\frac{24n^4 }{e^{n^3}}
\end{align*}
By l'H\^opital's rule, twice,
\begin{align*}
\lim_{x\rightarrow\infty} \frac{24x^4 }{e^{x^3}}
&=\lim_{x\rightarrow\infty} \frac{4\times 24x^3 }{3 x^2e^{x^3}}
&\text{by l'H\^opital} \\
&=\lim_{x\rightarrow\infty} \frac{32x }{e^{x^3}}
&\text{just cleaning up} \\
&=\lim_{x\rightarrow\infty} \frac{32 }{3x^2e^{x^3}}
&\text{by l'H\^opital, again} \\
&=0
\end{align*}
That's it. The limit comparison test now tells us that
$\sum_{n=1}^\infty a_n$ converges.

\end{description}

\noindent (b)
In part (a) we saw that $24n^2 e^{-n^3}$ is positive and decreasing.
The limit of this sequence equals $0$ (as can be shown with l'H\^opital's
Rule, just as we did at the end of the third solution of part (a)). Therefore, we can
use the alternating series test, so that the
error made in approximating the infinite sum
$S= \sum\limits_{n=1}^\infty a_n
= \sum\limits_{n=1}^\infty  (-1)^{n-1} 24n^2 e^{-n^3}$ by the sum of its first $N$ terms,
$S_N=\sum\limits_{n=1}^N a_n$,  lies between $0$
and the first omitted term, $a_{N+1}$.
If we use $5$ terms, the error satisfies
\begin{align*}
   |S - S_5 | \le |a_6| = 24 \times 36 e^{-6^3}\approx 1.3 \times 10^{-91}
\end{align*}



\end{solution}
%%%%%%%%%%%%%%%%%%%%%%%%%%%%%%%%

\begin{Mquestion}
You may assume without proof the following:
\[\sum_{n=0}^\infty \frac{(-1)^n}{(2n)!} = \cos(1)\]
Using this fact, approximate $\cos 1$ as a rational number, accurate to within $\frac1{1000}$.

Check your answer against a calculator's approximation of $\cos(1)$: what was your actual error?
\end{Mquestion}
\begin{hint}
You don't need to add up very many terms for this level of accuracy.
\end{hint}
\begin{answer}
$\cos 1 \approx \frac{389}{720}$; the actual associated error (using a calculator) is about
$0.000025$.
\end{answer}
\begin{solution}
The error in our approximation using through term $N$ is at most $\frac{1}{(2(N+1))!}$. We want $\frac{1}{(2(N+1))!}<\frac{1}{1000}$. By checking small values of $N$, we see that $8!=40320>1000$, so if $N=3$, then $\frac{1}{2(N+1)!}=\frac{1}{40320}<\frac{1}{1000}$. So, for our approximation, it suffices to consider the first four terms of our series.

\begin{align*}
\cos(2)&\approx \sum_{N=0}^3 \frac{(-1)^n}{(2n)!} = \frac{1}{0!}-\frac{1}{2!}+\frac{1}{4!}-\frac{1}{6!}\\
&=1-\frac12+\frac{1}{24}-\frac{1}{720}\\
&=\frac{720-360+30-1}{720}=\frac{379}{720}
\end{align*}
When we use a calculator, we see
\begin{align*}
\frac{389}{720}&=0.5402\overline{77} \\
\cos(1)&\approx  0.540302\\
\cos(1) - \frac{389}{720}&\approx 0.000024528\approx \frac{1}{40770}
\end{align*}
So, our error is reasonably close to our bound of $\frac{1}{40320}$, and far smaller than $\frac{1}{1000}$.
\end{solution}
%%%%%%%%%%%%%%%%%%%







%%%%%%%%%%%%%%%%%%%
\begin{Mquestion}
Let $a_n$ be defined as
\[a_n=\begin{cases}
-e^{n/2} & \text{ if $n$ is prime}\\
n^2 & \text{ if $n$ is \textbf{not} prime}
\end{cases}\]
Show that the series
$\displaystyle\sum_{n=1}^\infty\dfrac{a_n}{e^n}$  converges.
\end{Mquestion}
\begin{hint}
Use the direct comparison test to show that the series converges absolutely.
\end{hint}
\begin{answer}
See solution.
\end{answer}
\begin{solution}
The terms of this series are sometimes negative and sometimes positive. But, they are not strictly alternating, so we can't use the alternating series test. Instead, we use a direct comparison test to show the series converges absolutely.

If $n$ is prime, then
\[\left| \frac{a_n}{e^n}\right|=\left|-\frac{e^{n/2}}{e^n}\right|=\frac{1}{e^{n/2}}=\left(\frac{1}{\sqrt e}\right)^n\]

If $n$ is not prime, then
\[\left| \frac{a_n}{e^n}\right|=\left|-\frac{n^2}{e^n}\right|=\frac{n^2}{e^n}\]
For $n$ sufficiently large, $n^2<e^{n/2}$, so for $n$ sufficiently large,
\[\frac{n^2}{e^n}\le \left(\frac{1}{\sqrt{e}}\right)^n.\]

Since $e>1$, then $\sqrt{e}>1$, so the geometric series $\displaystyle\sum \left(\frac{1}{\sqrt{e}}\right)^n$ has $|r|=r=\frac{1}{\sqrt{e}}<1$, so it converges. By the direct comparison test, $\displaystyle\sum_{n=1}^\infty\left|\dfrac{a_n}{e^n}\right|$ converges as well. Then
$\displaystyle\sum_{n=1}^\infty\dfrac{a_n}{e^n}$ converges absolutely, so it converges.
\end{solution}


%%%%%%%%%%%%%%%%%%%
